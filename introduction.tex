\chapter*{Введение}
\addcontentsline{toc}{chapter}{Введение}  

	На ранних этапах развития интернета каждая web-страница доставлялась клиенту как статический документ, а интерактивность достигалась за счёт последовательности различных страниц. У такого подхода был недостаток~--- при каждом значительном изменении веб-страницы требовалось обратиться к серверу и перезагрузить эту страницу целиком. 

	В 1995 году компания Netscape выпустила язык программирования 
Java\-Script, который исполнялся на стороне клиента и позволял выполнять ряд задач, не обращаясь при этом к серверу. Вторым важным моментом было развитие AJAX-запросов~--- запросов к серверу, которые не требовали перезагрузки страницы. 
	
	Развитие этих двух технологий привело к появлению интернет-приложе\-ний~--- интерактивных web-сайтов, которые могут целиком располагаться на одной странице,  взаимодействуя с пользователем при помощи JavaScript кода, а с сервером при помощи AJAX запросов. На сегодняшний день такой формат сайтов очень популярен, т.к. он позволяет намного больше, чем последовательность интернет-страниц, например, писать браузерные игры, отображать карту и т.д.
	
	Одним из многочисленных типов интернет-приложений являются онлайн среды разработки~--- сайты, которые позволяют запустить код на том или ином языке в браузере. Они позволяют пользователям познакомиться с новым для них языком, а также предоставляют простой способ демонстрировать другим пользователям свой код (например, сайт jsfiddle почти всегда используется для публикации кода в качестве иллюстрации к ответам на вопросы, касающиеся HTML/CSS/JavaScript).
	
	Подобные приложения есть почти для всех распространённых языков программирования, например:
\begin{itemize}
	\item http://www.simplyscala.com/ - приложение, позволяющее запускать код на языке scala. 
	\item http://try.ceylon-lang.org/ - приложение, позволяющее запускать код на языке ceylon.
	\item http://www.tutorialspoint.com/ - приложение, позволяющее запускать код на большом количестве разнообразных языков программирования.
	\item \dots
\end{itemize}


	Онлайн-среда разработки для языка Котлин \cite{project_kotlin} называется Kotlin Web Demo. Данное приложение учитывает специфику  языка и позволяет как запускать код внутри виртуальной машины Java, которая располагается на сервере, так и транслировать код в JavaScript и исполнять его в браузере. Также в этом приложении есть подсветка синтаксиса, подсветка ошибок и автодополнение кода, что значительно упрощает процесс его написания.
	
	%TODO: разобраться со временами
	Недостатком данной среды разработки является исполнение всех запросов внутри одного сервера. Такая архитектура приложения является самой простой, однако при увеличении нагрузки приложение утрачивает работоспособность. Подобная ситуация наблюдалась при публикации Kotlin Web Demo и может повториться, например, при публикации новой версии Котлина.
	
	
	Вторым существенным недостатком была необходимость размещения  исполняемого кода  пользовательского примера внутри одного файла. Это не было проблемой само по себе, т.к. основная цель~--- дать пользователю познакомиться с языком, а не предоставить ему возможность писать большие программы, однако это стало проблемой, когда возникла идея использовать Kotlin Web Demo для запуска программ-заданий. Каждое такое задание требует от пользователя написать код на Котлине, который решает определённую задачу, а потом при помощи тестов проверяет, что код решает задачу корректно. Для реализации этого требовалось разделить код на файл с тестами и файл с пользовательским кодом, что и привело к возникновению вышеописанной проблемы.
	
	
	\underline{Целью} данного проекта является создание новой версии онлайн среды разработки.
	
	Основные \underline{задачи}, решаемые в работе:
	\begin{itemize}
		\item { \bf Создание автоматически масштабируемой архитектуры приложения} -- для работоспособности сервера в случае резкого увеличения нагрузки.
		\item { \bf Поддержка многофайловых проектов и JUnit тестов} -- для запуска программ-заданий.
	\end{itemize}
		


