\chapter*{Заключение}
\addcontentsline{toc}{chapter}{Заключение}
	В рамках данной работы было создано веб приложение, которое пришло на смену старому kotlin-demo.jetbrains.com и располагается по адресу try.kotlinlang.org. Данное приложение обладает следующими преимуществами по сравнению с предыдущей версией:
\begin{itemize}
	\item Обновлённый дизайн
	\item Улучшенная функциональность
	\item Обработка части запросов в облаке 
\end{itemize}
	
	Облачная часть приложения была протестирована, что позволяет с некоторой увереностью заявлять, что в случае возрастания нагрузки приложение продолжит стабильно работать.
	
	Была разработана сложная архитектура публикации приложения (см главу\ref{}), основной целью которой является создание удобного способа обновления приложения при изменении его исходного кода. Данная архитектура позволяет:
	
\begin{itemize}
	\item Проверить работоспособность обновлённого приложения перед его публикацией.
	\item Обновлять приложение не останавливая его работу.
	\item Иметь запасную версию приложения.
\end{itemize}

	В дальнейшем планируется использовать данное приложение в качестве платформы для обучения языку Котлин. Для этого будет использован имеющийся набор задач, который будет транслирован в набор проектов для Web Demo. На данный момент эти задачи представлены в виде проекта для IntellijIdea. По мере выполнения пользователем данных задач он сможет наблюдать свой прогресс.
	
	Также важным направлением развития является встраиваемость приложения в другие веб страницы. Одним из примеров использования встраиваемого приложения является публикация новостей о Котлине. При этом, как правило, выкладываются примеры кода, которые демонстрируют изменения языка. Данные примеры можно сделать исполняемыми, что позволит пользователю посмотреть на нововведения не переходя на основной сайт.