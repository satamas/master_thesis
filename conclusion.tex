\chapter*{Заключение}
\addcontentsline{toc}{chapter}{Заключение}
	В рамках данной работы была создана новая версия онлайн-среды разработки для языка Котлин. По сравнению с предыдущей версией, новое приложение обладает следующими преимуществами:
\begin{itemize}
	\item Поддержка многофайловых проектов
	\item Поддержка JUnit тестов
	\item Масштабируемая архитектура
\end{itemize}
	
	Масштабируемая часть приложения была протестирована. В рамках тестирования было оценено максимальное число пользователей, а также время восстановления системы после перегрузки.
	
	Была разработана инфраструктура (см. главу\ref{ch:infrastructure}), позволяющая обновить приложение при изменении его исходного кода, настроек приложения или требований к окружению. Данная схема обладает следующими преимуществами:
	
\begin{itemize}
	\item Возможность проверки работоспособности обновлённого приложения перед его публикацией
	\item Возможность обновления приложения без остановки его работы.
	\item Резервная версия приложения, которая может быть использована в случае выхода основной версии из строя.
\end{itemize}

	Был автоматизирован процесс настройки окружения, а также процессы публикации приложения и его обновления.
	
\section*{Направления развития}

	В дальнейшем планируется использовать данное приложение в качестве платформы для обучения языку Котлин. Для этого будет использован имеющийся набор задач, который будет транслирован в набор проектов для Web Demo. На данный момент эти задачи представлены в виде проекта для IntellijIdea. По мере выполнения пользователем данных задач он сможет наблюдать свой прогресс.
	
	Также важным направлением развития является встраиваемость приложения в другие web-страницы. Одним из примеров использования встраиваемого приложения является публикация новостей о Котлине. При этом, как правило, выкладываются примеры кода, которые демонстрируют изменения языка. Данные примеры можно сделать исполняемыми, что позволит пользователю посмотреть на нововведения, не переходя при этом на сайт онлайн-среды разработки.