\chapter{Устройство приложения}
Моё приложение состоит из серверной и клиентской части. Серверная часть в свою очередь состоит из фронтенд сервера - приложения, которое принимает запросы от пользователя и бэкенд серверов, которые обрабатывают ряд запросов, пришедших на фронтенд сервер.


	
	%Написать про базу данных
	%Добавить картинку
	%Обновление бэкенда при обновлении версии Котлина
	%Сказать про JS конфигурацию
\section{Серверная часть}
\subsection{Технологии}
	Для написания серверной части используются следующие технологии:
\begin{itemize}
	\item \textbf{Tomcat} -- сервер, написанный на языке Java и разрабатываемый компанией Apache. Томкат реализует ряд спецификаций языка Java, таких как Java Servlet и JavaServer Pages. Для обработки запросов, приходящих на сервер, томкат использует отдельные потоки, количество которых может регулироваться в настройках сервера. 
	
	Используется как на фронтенд сервере, так и на бэкенд серверах.
	\item \textbf{MySql} -- реляционная база данных. 
	
	Используется в нашем приложении для хранения сохранённых пользовательских проектов.
\end{itemize}


\subsection{Фронтенд сервер}
	Фронтенд сервер принимает все запросы, приходящие от пользователя. Эти запросы делятся на несколько типов:
\begin{itemize}
	\item Запросы, связанные с Котлином (исполнение, подсветка синтаксиса, автодополнение, конвертация из Java, конвертация в JS)
	\item Запросы  к базе данных (добавить/модифицировать/сохранить/удалить проект/файл)
	\item Запросы к фронтенд серверу (выдать код примера, запросы авторизации)
\end{itemize}

	Все запросы связанные с Котлином пересылаются бэкенд серверам, после чего в течении определённого времени ожидается ответ, который отправляется обратно пользователю. Если же ответ не приходит, то пользователю отправляется соответствующее сообщение об ошибке.
	
	Запросы к базе данных отправляются только для авторизированных пользователей, у которых в базе дынных хранятся их программы. Авторизироваться на данный момент можно используя свой аккаунт google, facebook, twitter, github или JetBrains account. В большинстве случаев для авторизации используется протокол OAuth. После того как пользователь авторизировался, его данные (идентификатор, имя, фамилия) записываются внутрь HTTP сессии. 

\subsection{Бэкнед сервер}
	Каждый бэкенд сервер, так же как и фронтенд сервера, представляет собой приложение, запущенное внутри tomcat. 
	
	В качестве библиотеки к данному приложению подключена релизная версия всех библиотек Котлина (runtime, compiler, plugin). Эти библиотеки используются для обработки всех пользовательских запросов.
	
	На данный момент вся обработка запроса, кроме исполнения скомпилированной в java программы, осуществляется в потоке сервера. Это приводит к тому, что мы не можем гарантированно ограничить время исполнения запроса, т.к. поток нельзя завершить извне.
	
	При обработке запроса на исполнение сначала код на языке Котлин компилируется в байткод java. Далее скомпилированный код исполняется в отдельном процессе внутри класса - обёртки. Этот класс выбирается в зависимости от конфигурации проекта, отправленного нам на исполнение.
	
	В случае JVM конфигурации этот класс, используя reflection, запускает функцию main в пользовательском коде, т.е. по большому счёту, просто исполняет пользовательскую программу. При этом весь вывод программы (stdOut, stdErr) перенаправляется в форматированном виде в один поток. Если внутри пользовательского кода вылетает ошибка, она ловится классом обёрткой. По завершению пользовательской программы весь вывод, а так же вылетевшая ошибка (если есть), сериализуются в формат JSON и полученная строчка выводится в stdout класса - обёртки.
	
	В случае Junit конфигурации, используя встроенные средства этой библиотеки, запускаются все тестовые методы, найденные внутри пользовательских классов. При запуске каждого теста сохраняется вывод программы и ошибки, вылетевшие в процессе исполнения. По завершению исполнения всех тестов эта информация так же сериализуется в формат JSON и выводится в стандратный поток вывода.
	
	Процесс, в котором исполняется пользовательская программа, ограничен по памяти (Xmx настройка java-машины), ограничен по времени исполнения (через 5 секунд процесс завершается из сервера), а так же ему запрещён доступ к сети, файловой системе и.т.д. при помощи Java security manager.
	
	
	